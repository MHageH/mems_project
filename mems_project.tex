\documentclass[a4paper]{article}
%\usepackage[T1]{fontenc}
\usepackage[english]{babel}

\usepackage{amsmath}
\usepackage{amssymb,amsfonts,textcomp, graphicx}
\usepackage{graphics}

\usepackage{wrapfig}

\usepackage{parskip}

\usepackage{color}
\usepackage{array}
\usepackage{hhline}
\usepackage{subcaption}

\usepackage{textcomp}

\usepackage[hidelinks]{hyperref}

\setlength\tabcolsep{1mm}
\renewcommand\arraystretch{1.3}

\setlength\voffset{-1in}
\setlength\hoffset{-1in}
\setlength\topmargin{0.7874in}
\setlength\oddsidemargin{0.7874in}
\setlength\textheight{10.118099in}
\setlength\textwidth{6.6932993in}
\setlength\footskip{0.0cm}
\setlength\headheight{0cm}
\setlength\headsep{0cm}

% Code specific

\usepackage{listings}

\definecolor{mygreen}{rgb}{0,0.6,0}
\definecolor{mygray}{rgb}{0.5,0.5,0.5}
\definecolor{mymauve}{rgb}{0.58,0,0.82}

\lstset{ %
  backgroundcolor=\color{white},   % choose the background color; you must add \usepackage{color} or \usepackage{xcolor}
  basicstyle=\footnotesize,        % the size of the fonts that are used for the code
  breakatwhitespace=false,         % sets if automatic breaks should only happen at whitespace
  breaklines=true,                 % sets automatic line breaking
  captionpos=b,                    % sets the caption-position to bottom
  commentstyle=\color{mygreen},    % comment style
  deletekeywords={...},            % if you want to delete keywords from the given language
  escapeinside={\%*}{*)},          % if you want to add LaTeX within your code
  extendedchars=true,              % lets you use non-ASCII characters; for 8-bits encodings only, does not work with UTF-8
  frame=single,	                   % adds a frame around the code
  keepspaces=true,                 % keeps spaces in text, useful for keeping indentation of code (possibly needs columns=flexible)
  keywordstyle=\color{blue},       % keyword style
  language=Octave,                 % the language of the code
  otherkeywords={*,...},           % if you want to add more keywords to the set
  numbers=left,                    % where to put the line-numbers; possible values are (none, left, right)
  numbersep=5pt,                   % how far the line-numbers are from the code
  numberstyle=\tiny\color{mygray}, % the style that is used for the line-numbers
  rulecolor=\color{black},         % if not set, the frame-color may be changed on line-breaks within not-black text (e.g. comments (green here))
  showspaces=false,                % show spaces everywhere adding particular underscores; it overrides 'showstringspaces'
  showstringspaces=false,          % underline spaces within strings only
  showtabs=false,                  % show tabs within strings adding particular underscores
  stepnumber=2,                    % the step between two line-numbers. If it's 1, each line will be numbered
  stringstyle=\color{mymauve},     % string literal style
  tabsize=2,	                   % sets default tabsize to 2 spaces
  title=\lstname                   % show the filename of files included with \lstinputlisting; also try caption instead of title
}

%


\begin{document}

\newcommand\textstyleEmphasis[1]{\textit{#1}}
\renewcommand{\contentsname}{Table des mati\`eres}
\renewcommand\refname{R\'ef\'erences}

\renewcommand{\abstractname}{Pr\'eambule}
\title{\textbf{Projet Conception de Microsyst\`emes \\ Altim\`etre int\'egr\'e en VHDL-AMS}}
\author{Mohamed Hage Hassan \\ Cl\'ement Cheung}
\date{8 Decembre 2017}
\maketitle
\thispagestyle{empty}

\tableofcontents
\clearpage

\section{Introduction}

\section{Architecture G\'en\'erale de l'altim\`etre}
\section{Mod\'elisation et simulation des blocs}

Test $\{ Verilog \}$
\begin{lstlisting}[language=Verilog, belowskip=-0.5 \baselineskip]
module thermo2bin (thermob, bin)
input[62:0] thermob;
output [5:0] bin;

reg [62:0] thermo;
reg [5:0] bin, bin1, bin2;
integer i, j, k;

always @(thermob)
begin
  for (k = 0; k <=60; k=k+1)
    thermo[k] <= thermob[k] || thermob[k+1] || thermob[k+2];

  thermo[61] <= thermob[61] || thermob[62];
  thermo[62] <= thermob[62];
end

always @(thermo)
begin
  bin1 = 0;
  for(i=1; i <= 32; i=i+1)
    if (thermo[i-1] = 1'b1) bin1 = i;
end

always @(thermo)
begin
  bin2 = 0;
  for(j=1; j<=31; j=j+1)
    if(thermo[k+31] == 1'b1) bin2 = j;
end

always @(bin1 or bin2)
if (thermo[31] == 1'b1)
  bin = bin2 + 32;
else
  bin = bin1;

endmodule

\end{lstlisting}

\subsection{Transducteurs comme elements de test}
\subsection{Capteur de temp\'erature}
\subsection{Architecture de l'amplificateur diff\'erentiel}
\subsection{Convertisseur Analogique-Num\'erique en VHDL-AMS}
\subsection{Logique de commande}


\clearpage

\section{Conclusion}

\clearpage
\addcontentsline{toc}{section}{R\'ef\'erences}

\begin{thebibliography}{9}

\bibitem{sim-elec-cours}
\textit{Simulation \'electromagn\'etique et techniques de mesure RF,}\\
\texttt{Jean-Daniel Arnould, Institut Polytechnique de Grenoble - Phelma}

\end{thebibliography}


\end{document}
